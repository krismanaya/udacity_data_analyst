\documentclass[8pt]{article}
\usepackage{xcolor}
\usepackage{hyperref}
\hypersetup{urlcolor=blue}
\usepackage{times}
\usepackage{pstricks}
\usepackage{listings}
\usepackage{enumerate}
\usepackage{pst-all}
\usepackage{color}
\usepackage{pstcol,pst-node,pst-coil}
\usepackage[dvips]{graphics}
\usepackage{amsfonts,amsmath,amsthm,amssymb,euscript,amscd}
\usepackage[english]{babel}
\usepackage[autostyle]{csquotes}
\usepackage[all,cmtip]{xy}
\usepackage{float}
\usepackage{pdfpages}
\usepackage{graphicx} 
\pagestyle{headings}
\setlength{\parindent}{3ex}
\setlength{\parskip}{1.5ex plus0.5ex minus 0.5ex}
\setlength{\textwidth}{14cm}
\setlength{\oddsidemargin}{1cm}
\setlength{\evensidemargin}{1cm}
\setlength{\textheight}{20.5cm}
\newcommand{\newln}{\\&\quad\quad{}}
\newcommand{\parenthnewln}{\right.\\&\left.\quad\quad{}}
\newcommand{\sub}{\subset} 
\newcommand{\T}{\mathcal{T}}
\newcommand{\R}{\mathbb{R}} 
\newcommand{\U}{\mathcal{U}}
\newcommand{\B}{\mathcal{B}}
\newcommand{\C}{\mathcal{C}}
\newcommand{\8}{\bar}
\newcommand{\maps}{\mapsto}
\newcommand{\norm}[1]{\left\lVert#1\right\rVert}
\newcommand{\e}{\epsilon}
\newcommand{\del}{\delta}
\newcommand{\f}{\frac}
\newcommand{\ol}{\overline}
\newcommand{\la}{\langle}
\newcommand{\ra}{\rangle}
\newcommand{\Par}{\partial}
\renewcommand{\qedsymbol}{\rule{0.7em}{0.7em}}
\newtheorem{theorem}{Theorem}[section]
\newtheorem{lemma}[theorem]{Lemma}
\newtheorem{proposition}[theorem]{Proposition}
\newtheorem{corollary}[theorem]{Corollary}


\begin{document} 
\title{The Science of Decisions} 
\author{\textit{Kris Anaya}}
\date{\today} 
\maketitle

\begin{abstract} 
\textit{In a Stroop task, participants are presented with a list of words, with each word displayed in a color of ink. The participant's task is to say out loud the color of the ink in which the word is printed. The task has two conditions: a congruent words condition, and an incongruent words condition. In each case, we measure the time $\textbf{t}$, it takes to name the ink colors in equally-sized lists. Each participant will go through and record a time from each condition.} 
\end{abstract}
\section{Introduction}
In this task, we are asked to measure the time it takes to name the ink colors in equally-sized lists. In the \textit{congruent} words, the words that are displayed are color words whose names match the colors in which they are printed. In the \textit{incongruent} words conditions, the words displayed are color words whose names do no match the colors in which they are printed. Meaning, what is the probability it takes our independent variable $t$ to name the colors in the \textit{congruent} and \textit{incongruent} words within these two independent samples which are dependent on $t$.\textbf{\textit{Within this project we test that there is no significant change in speed between the  \textit{congruent} and \textit{incongruent} samples. However, this will be tested against alternative changes between our \textit{congruent} and \textit{incongruent} samples given our independent variable $t$ at $\alpha = 0.05$}}: \newline \newline
$H_{0} :=$ \textbf{\textit{are there no changes between \textit{congruent} and \textit{incongruent} samples}} \newline
$H_{A} :=$ \textbf{\textit{are there significant changes between \textit{congruent} and \textit{incongruent} samples.}} 
\begin{equation}
\begin{split}
H_{0}: \mu_{c} = \mu{I}  \\
H_{A}: \mu_{c} \not= \mu{I}
\end{split}
\end{equation}
We will perform these task by using a Pythonic language module which we have created. This module is aptly named \textit{statistics.py}. 
The resources for our code may be found within our GitHub page \textit{http://github.com/krismanaya} under \textit{UdactiyDataScience}. \textbf{\textit{The test we shall be focused on when comparing the means will be to perform a $t$-test, since we assume that both means are  positively skewed means. These assumptions will be made within our visualizations in the results section of our report.}} Within our methods, we may conclude that a Type I error is very unlikely. Therefore, the test procedure will be based on a rejection \textbf{$H_{0}$ at an $\alpha$ level of $0.05$}. In conclusion, we will show there is a significant difference between naming \textit{congruent} and \textit{incongruent} colors by relying on our independent variable $t$. 
\newpage
\section{Methods}
\subsection{sample} 
The sample population of interest is taken from the Stroop dataset provide by the Udacity Class. This data set may be viewed on Udacity's Google drive \href{https://drive.google.com/file/d/0B9Yf01UaIbUgQXpYb2NhZ29yX1U/view}{website}. The sample population that is provided consists of a list of patients who were tested twice, one saying the words with applying the \textit{congruent} method and then tested again saying the word with the \text{incongruent} method. Both samples were done under $t$ constraint. 
\subsection{design} 
\textbf{\textit{A repeated measures longitude two-test randomized experimental design was provided for this study. We wish to analyze the effect size of the given treatments. The given variables  and measures will be the mean, size, standard-error, t-statistics, confidence interval; degrees of freedom, t-critical and Bessel's approximation.These measures and variables will be denoted as:}}
\newline
\newline
\newline

\fbox{\parbox{\textwidth}{\textit{Stroop variables:}
\begin{enumerate}
\item $\bar{X}_{c} :=$ \textbf{This mathematical symbol represents the congruent sample mean} 
\item $\bar{X}_{I} :=$ \textbf{This mathematical symbol represents the incongruent sample mean}
\item $\bar{X}_{d} :=$ \textbf{This is the mean difference or $\bar{X}_{I} - \bar{X}_{c}$} 
\item $n_{c,i} := $  \textbf{This mathematical symbol represents the size of the our sample} 
\item $\alpha :=$ \textbf{this symbol represents our alpha level that we are choosing to test}
\item $t_{c} :=$ \textbf{this mathematical symbol represents the critical value of the t score}
\item $t_{stat} :=$ \textbf{this mathematical symbol represents the statistical value of our found t} 
\item $df :=$ \textbf{this is the degrees of freedom or, $t_{n-1}$ score} 
\item $SE :=$ \textbf{this mathematical symbol is the Standard Error} 
\item $\sigma_{c,i} :=$ \textbf{This mathematical symbol is the standard deviation} 
\end{enumerate}}}
\newline
\newline
Within this design we test that there is no significant change in speed between the \textit{congruent} and \textit{incongruent} samples. However, this study will be tested against alternative changes between our $H_{0}$ given our independent variable $t$ at $\alpha = 0.05$: 
\begin{equation}
\begin{split}
H_{0}: \mu_{c}-\mu_{I} = 0 \\
H_{A}: \mu_{c}-\mu_{I} \not= 0 
\end{split}
\end{equation}
\newpage
\section{Results} 
\subsection{variability} 
Within the results section we round all data points two decimals places. We begin by locating the variables, we used our Pythonic module \textit{statistics.py} to create a two-list aptly named \text{congruent} and \textit{incongruent}. \textbf{\textit{Secondly, we calculated the mean $\bar{X}_{c,i}$, $\sigma_{c,i}$ and $\bar{X}_{d}$ with help of our functions}}: 
\begin{table}[htbp]\centering \caption{Stroop statistics \label{sumstat}}
\begin{tabular}{l c c  }\hline\hline
\multicolumn{1}{c}{\textbf{Variable}} & $\bar{X}$ & $\sigma$ \\ \hline
Congruent & 14.05 & 3.56  \\
Incongruent & 22.02 & 4.80  \\
difference & 7.96 & 4.86 \\ 
\multicolumn{1}{c}{n} & \multicolumn{2}{c}{24}\\ \hline
\end{tabular}
\end{table}
\small{\begin{verbatim}
congruent = make_a_list() 
incongruent = make_a_list() 
d = difference(incongruent,congruent) 
x_c = mean(congruent) 
>>> 14.05
x_i = mean(incongruent)
>>> 22.02 
x_d = mean(d) 
>>> 7.96
sigma_c = bessels_correction(congruent) 
>>> 3.56
sigma_i = bessels_correction(incongruent) 
>>> 4.80 
sigma_d = bessels_correction(d) 
>> 4.86 
\end{verbatim}}
\noindent\textbf{\textit{The difference $d$ was calculated by taking $d = \bar{X}_{I} - \bar{X}_{c}$ and then taking its mean $\bar{d}$ this gives us a second experiment to try during out t-test. So we may calculate the $t_{n-1}$ table at $n_{c-i} = 26$ and $n_{d} = 23$ respectfully. This will show without a doubt that we will reject our $H_{0}$ hypothesis.}} The next section will provide visualization of our assumptions of the two lists with a histogram and box plot. 

\subsection{Histogram/Boxplot} 
\begin{figure}[H]
	\includegraphics{histogram.png}
\end{figure}
\begin{figure}[H]
	\includegraphics{box_plot.png}
\end{figure}


\noindent \textbf{\textit{The reader may have observed that within the mean and standard deviation, the length and average $t$ is much smaller for our congruent list of data points. This may suggest to us that the first test has an overall faster time of saying words. Secondly, from the histogram image above we see that these lists are positively skewed making it a more likely case to use point estimation. Finally, from the box plot we that the median are not very close to zero except maybe the difference , there are limited outlies and the distribution of the data seems similar to what we have in our code shown above.}} 
\subsection{central tendencies} 
Within our central tendencies we wish to calculate the standard error, t-statistic, degrees of freedom and t-critical at $\alpha = 0.05$: 
\begin{table}[htbp]\centering \caption{Stroop statistics \label{sumstat}}
\begin{tabular}{l c c  }\hline\hline
\multicolumn{1}{c}{\textbf{Central Tendencies}} & $X_{c} - X_{i}$ & $X_{d}$ \\ \hline
$SE$ & 1.22 & 0.99\\
$df$  & 46 & 23\\
$t_{stat}$ & -6.53 & 8.020 \\ 
$t_{c}$ &  2.009 &  2.068\\ 
$\alpha$ & 0.05 & 0.05\\
\multicolumn{1}{c}{-} & \multicolumn{2}{c}{-}\\ \hline
\end{tabular}
\end{table}
\begin{verbatim}
SE = independent_standard_error(congruent,incongruent,len(congruent),len(incongruent)) 
>>> 1.22 
SE_d = standard_sample_error(d,len(d)) 
>>> 0.99
df = degrees_of_freedom_indy(congruent,incongruent) 
>>>  ((24, 24), 46)
df_d = len(d) - 1 
>>> 23 
t_stat = t_stat_indy(x_c,x_i,0,0,sigma_c,sigma_i,24,24)
>>> -6.53 
t_stat_d =  mean(d)/SE 
>>> 8.020 
# t_c = use chart* 
\end{verbatim}
\textbf{\textit{Here we visualize a t-distribution at $\alpha$ level denoted above and at $0.95$ confidence interval. As we can see our $t_{stat}$ variable is well below the critical value. This suggests that it is possible there is some significant changes between the \textit{congruent} and \textit{incongruent}. Because of this we shall reject $H_{0}$ on both cases where our $t_{stat}$ score falls way below both tails of the critical level. and conclude that there exist a probability $P < \alpha$.}}\newline\newline 

\subsection{t-distribution} 
\textbf{\textit{As seen below we have shown that both out $t_{stats}$ for our tests are  $-6.53$ and $8.020$ at a $\alpha$ level of $.95$. We can conclude from looking at these two distributions below that we reject our $H_{0}$ and conclude that there are significant changes between congruent and incongruent tests. This is what we conclude for the end of our discussion.}}
\begin{figure}[H]
\includegraphics[page=1,scale=0.8]{t-distro.pdf}
\end{figure}

\begin{figure}[H]
\includegraphics[page=1,scale=0.8]{t_stat_d.pdf} 
\end{figure} 

\section{Discussion}
The effects responsible for a rejection of $H_{0}$ could be due to the patient's eyes becoming comfortable with words and colors that when they are ready to observe the \textit{incongruent} test their eyes have still not settled. One test you could do is switch the test and perform a Stroop task of \textit{incongruent} then \textit{congruent} and see if the samples differ. Another similar approach for the test would make the patient perform the task multiple times to see if there is some level of improvement after practicing the \textit{incongruent} tests. Finally, a last possibility could be age and eyesight. Do these patients are different in age are they similar? What are the patients eyesight, is it normal or do some patients struggle with color? We do not have more information within the data set. The only information given to us is the independent variable $t$. 
\end{document}